\documentclass[11pt]{article}

\usepackage[a4paper,left=1cm,right=1cm,top=1.2cm,bottom=1.5cm]{geometry}
\usepackage{xcolor}
\usepackage{enumitem}
\usepackage{xeCJK}
\setCJKmainfont{Noto Serif CJK SC}
\usepackage{fontawesome}
\usepackage{libertine}

\newcommand{\subsec}[1]{
  \vskip 4pt
  {\bf\large\color{teal}  #1}
  \vskip 1pt
}
\newcommand{\yearspaceevent}[3]{
  {\color{gray} \small #1 \hfill #2}\\
  {\bf #3}
}
\newcommand{\yearat}[1]{
  {\hfill \color{gray} #1}
}

\setlist{leftmargin=4mm,topsep=3pt,itemsep=-1pt}

\begin{document}
\pagenumbering{gobble}

{
  \centering {\bf\large 郑帆} \\
  \faEnvelopeO \space hi@fzheng.me \space\space
  \faLink \space  fzheng.me \space \space 
  \faGithub \space github.com/izhengfan
\par}
\noindent
\begin{minipage}[t]{0.6\textwidth}

  \subsec{工作经历}

  \yearspaceevent{2019-}{深圳 }{未来机器人,研发经理}
  \begin{itemize}
    \item AGV 导航
  \end{itemize}

  \yearspaceevent{2015-2016}{深圳 \& 香港}{未来机器人,SLAM 实习生}
  \begin{itemize}
    \item 开发用于工业室内环境 AGV 定位的融合单目视觉与里程计信息的实时 SLAM 系统
    \item 定位精度达到 \~{}10cm 级别
  \end{itemize}

  \yearspaceevent{2013}{无锡}{阿特拉斯·科普柯(无锡),机器人实习生}
  \begin{itemize}
    \item 开发用于空气压缩机转子边缘去毛刺的六自由度机械臂原型
    \item 实现了一个原型,末端执行器可按预设轨迹运动
  \end{itemize}

  \subsec{项目经历}
  \yearspaceevent{2012-2014}{杭州}{仿人机器人机械设计}
  \begin{itemize}
    \item 团队为 \emph{ZJUDANCER},开发仿人型足球机器人
    \item 设计机器人的机械结构,在 SolidWorks 中建立三维模型,绘制加工图纸并联系厂家加工零件,装配并维护机器人硬件
    \item 2013-2014 年间为机械组负责人
  \end{itemize}
  \yearspaceevent{2013-2014}{杭州}{触觉传感器测试平台开发}
  \begin{itemize}
    \item 毕业设计项目,包含硬件和软件开发
    \item 开发一个实现三维平移和三维转动的机电平台,能精准地驱动探针对触觉传感器完成预设的测试刺激动作
  \end{itemize}
  \yearspaceevent{2014-2019}{香港}{里程计/IMU 辅助视觉状态估计}
  \begin{itemize}
    \item 博士论文研究课题
    \item 开发用于地面车辆的里程计/IMU 辅助视觉状态估计系统,挖掘利用运动先验信息,在图优化框架中实现
    \item 实现了比主流前沿的视觉估计或视觉惯性估计系统更好的精度和鲁棒性
  \end{itemize}

  \subsec{发表论文}

  \begin{itemize}
    \item {\bf Fan Zheng}, Yun-Hui Liu. ``SE(2)-Constrained Visual Inertial Fusion for Ground Vehicles''. {\it IEEE Sensors Journal}, vol. 18, no. 23, 2018.
    \item {\bf Fan Zheng}, Hengbo Tang, Yun-Hui Liu. ``Odometry-Vision-Based Ground Vehicle Motion Estimation With SE(2)-Constrained SE(3) Poses''. {\it IEEE Transactions on Cybernetics}, vol. 49, no. 7, 2019.
    \item {\bf Fan Zheng}, Yun-Hui Liu. ``Visual-Odometric Localization and Mapping for Ground Vehicles Using SE(2)-XYZ Constraints''. {\it Proc. IEEE International Conference on Robotics and Automation (ICRA)}, 2019.
    \item {\bf Fan Zheng}, Yun-Hui Liu. ``Keypoint Matching Outlier Removal with 3DMP Histogram Voting''. {\it Proc. World Congress on Intelligent Control and Automation (WCICA)}, 2018.
    \item {\bf Fan Zheng}, Yun-Hui Liu. ``A Geometric Model for Fusing IMU into Monocular Visual Localization of 3-D Mobile Robots''. {\it Proc. IEEE International Conference on Real-time Computing and Robotics (RCAR)}, 2016.
  \end{itemize}

\end{minipage}
\hfill\begingroup\color{teal}\vrule width 1pt\endgroup\hfill
\noindent
\begin{minipage}[t]{0.34\textwidth}
  \subsec{教育背景}
  \yearspaceevent{2014-2019}{香港}{香港中文大学}
  \begin{itemize}
    \item 博士,机械与自动化工程
  \end{itemize}

  \yearspaceevent{2010-2014}{杭州}{浙江大学}
  \begin{itemize}
    \item 学士,机械电子工程
    \item 辅修,晨兴文化中国人才计划
  \end{itemize}

  \subsec{教学经历}
  \emph{作为助教}
  \begin{itemize}
    \item 机器人学 \yearat{2015/2016}
    \item 现代控制与理论 \yearat{2014/2015}
    \item 材料力学 \yearat{2014/2015}
  \end{itemize}

  \subsec{志愿服务}
  \begin{itemize}
    \item 学生大使,香港中文大学工程部 HKPFS 夏令营  \yearat{2015}
    \item 志愿者,IEEE ROBIO 国际会议 \yearat{2014}
    \item 负责人,晨兴文化中国人才计划四期毕业典礼 \yearat{2014}
  \end{itemize}

  \subsec{技能}
  {\bf 语言}
  \begin{itemize}
    \item 普通话,潮州话(母语)
    \item 英语,粤语(流利)
  \end{itemize}
  {\bf 编程}
  \begin{itemize}
    \item C++(Qt, ROS, OpenCV, Eigen, g2o, Ceres Solver)
    \item Python
    \item MATLAB
  \end{itemize}
  {\bf 软件}
  \begin{itemize}
    \item MS Office
    \item SolidWorks, AutoCAD
    \item \LaTeX
    \item Bash, Git, Vim, CMake 等
  \end{itemize}

  \subsec{荣誉}
  \begin{itemize}
    \item Hong Kong PhD Fellowship (每年授予在港 200 余博士生) \yearat{2014}
    \item RoboCup 国际赛小仿人组八强 \yearat{2014}
    \item RoboCup 中国赛小仿人组冠军 \yearat{2013}
    \item 国家奖学金(1/72) \yearat{2012}
  \end{itemize}
\end{minipage}
\end{document}
